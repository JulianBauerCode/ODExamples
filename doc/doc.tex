\documentclass[10pt,a4paper]{article}
\usepackage[utf8]{inputenc}
\usepackage[english]{babel}
\usepackage{amsmath}
\usepackage{amsfonts}
\usepackage{amssymb}
\usepackage{graphicx}
\graphicspath{{./pictures/}}
\usepackage[colorlinks=true,linkcolor=black,urlcolor=blue,bookmarksopen=true]{hyperref}
\usepackage{placeins}
\usepackage{listings}
\usepackage{siunitx}

\usepackage[left=3cm, right=3cm, top=3cm]{geometry}

\usepackage{fancyhdr}	% Fancy footer and header
\usepackage{lastpage}	% Get total number of pages

\pagestyle{fancy}
\fancyhf{}
\rhead{}
\lhead{}
\rfoot{\thepage{} von \pageref{LastPage}}	%https://tex.stackexchange.com/a/32537

\author{Julian Bauer}
\title{ODExamples}

\include{00_macro}

%Avoid indentation of new paragraphs
\setlength{\parindent}{0pt}

%%Add 4-th level of sections
% Adjust formattign of paragraph
%\makeatletter
%\renewcommand\paragraph{\@startsection{paragraph}{4}{\z@}%
%            {-2.5ex\@plus -1ex \@minus -.25ex}%
%            {1.25ex \@plus .25ex}%
%            {\normalfont\normalsize\bfseries}}
%\makeatother
%\setcounter{secnumdepth}{4} % how many sectioning levels to assign numbers to
%\setcounter{tocdepth}{4}    % how many sectioning levels to show in ToC
%\newcommand{\subsubsubsection}[1]{\paragraph{#1}\mbox{}\\}





\begin{document}
\definecolor{codegreen}{rgb}{0,0.6,0}
\definecolor{codegray}{rgb}{0.5,0.5,0.5}
\definecolor{codepurple}{rgb}{0.58,0,0.82}
\definecolor{backcolour}{rgb}{0.95,0.95,0.92}
\lstdefinestyle{mystyle}{
    backgroundcolor=\color{backcolour},   
    commentstyle=\color{codegreen},
    keywordstyle=\color{magenta},
    numberstyle=\tiny\color{codegray},
    stringstyle=\color{codepurple},
    basicstyle=\footnotesize,
    breakatwhitespace=false,         
    breaklines=true,                 
    captionpos=b,                    
    keepspaces=true,                 
    numbers=left,                    
    numbersep=5pt,                  
    showspaces=false,                
    showstringspaces=false,
    showtabs=false,                  
    tabsize=2
}
\lstset{style=mystyle}


\maketitle
\tableofcontents
\newpage

\section{Problems}

\subsection{Reaction kinetics}
Source: \cite{Fritzen2013}

\subsubsection{Problem statement}
\begin{align}
\dot{\vu}
=
\begin{bmatrix}
\dot{u}_0\\
\dot{u}_1\\
\dot{u}_2
\end{bmatrix}
=
\underbrace{
	k_0
	u_u
	\begin{bmatrix}
	-1\\
	1\\
	0
	\end{bmatrix}
}_{term 0}
+
\underbrace{
	k_1
	u_1
	u_2
	\begin{bmatrix}
	1\\
	-1\\
	0
	\end{bmatrix}
}_{term 1}
+
\underbrace{
	k_2
	u_1
	\begin{bmatrix}
	0\\
	-1\\
	1
	\end{bmatrix}
}_{term 2}			\label{eq:reactionKinetics}
\end{align}

\begin{table}[!h]
\centering
\begin{tabular}{l|l}
Symbol 		& Interpretation with $i \in [0,1,2]$\\ \hline
$u_i$		& Contentration of particles of type $i$\\
$k_i$		& Constant of proportionallity of term $i$\\
$\dot{u}_i$ & Rate of change of $u_i$
\end{tabular}
\caption{Symbols in \autoref{eq:reactionKinetics}}
\label{tab:01_symbols}
\end{table}

\subsubsection{Interpretation}

\begin{figure}[!h]
\centering
\includegraphics[width=0.4\textwidth]{01_reactionKinetics}
\caption{Visualization of \autoref{eq:reactionKinetics}}
\label{fig:01_vis}
\end{figure}

\paragraph*{Term0}
Particles of type $0$ turn into particles of type $1$,
decreasing the concentration $u_0$ and increasing $u_1$.
The rate of change is proportional to the concentration $u_0$ with the
constant of proportionallity $k_0$.

\paragraph*{Term1}
Particles of type $1$ turn into particles of type $0$,
decreasing the concentration $u_1$ and increasing $u_0$.
The rate of change is proportional to the product of the concentrations
$u_0$ and $u_1$ with the constant of proportionallity $k_1$.
This means, the presence of type $2$ particles catalyze the transition of type $1$ particels into type $0$ particles.
Term0 and Term1 counteract each other.

\paragraph*{Term2}
Particles of type $1$ turn into particles of type $2$,
decreasing the concentration $u_1$ and increasing $u_2$.
The rate of change is proportional to the concentration $u_1$ with the
constant of proportionallity $k_2$.

\section{Python}

\subsection{Code}

\paragraph{Note:}
Symbols of concentrations $u_i$ are replaced by $y_i$.

\lstinputlisting[language=Python]{../01_reactionKinetics/reactionKinetics.py}

\subsection{Pictures}

\begin{figure}[!h]
\centering
\includegraphics[width=\textwidth]{../01_reactionKinetics/simple}
\end{figure}

\begin{figure}[!h]
\centering
\includegraphics[width=\textwidth]{../01_reactionKinetics/influence_method}
\end{figure}

\begin{figure}[!h]
\centering
\includegraphics[width=\textwidth]{../01_reactionKinetics/influence_k}
\end{figure}

\begin{figure}[!h]
\centering
\includegraphics[width=\textwidth]{../01_reactionKinetics/influence_y0}
\end{figure}

\FloatBarrier
\newpage
\listoffigures
\listoftables
\lstlistoflistings
\bibliographystyle{./my_elsart-harv_engl}
\bibliography{./bib}
\end{document}
